\documentclass[12pt]{article}
\usepackage{amsmath,amssymb}
\usepackage{fancybox}\usepackage{epsfig}\usepackage{psfig}
%\usepackage{showkeys}
\textwidth 6.5in\oddsidemargin 0in
\textheight 9in\topmargin -0.5in
\renewcommand{\baselinestretch}{1.01}
\newtheorem{thm}{Theorem}[section]
\newtheorem{cor}[thm]{Corollary}
\newtheorem{lem}[thm]{Lemma}
\newtheorem{conj}[thm]{Conjecture}
\newtheorem{assumption}[thm]{Assumption}
\newenvironment{assum}{\begin{assumption}\rm}{\end{assumption}}
\newtheorem{prop}[thm]{Proposition}
\newtheorem{crit}[thm]{Criterion}
\newtheorem{prob}[thm]{Problem}
\newtheorem{alg}[thm]{Algorithm}
\newtheorem{definition}[thm]{Definition}
\newenvironment{defn}{\begin{definition}\rm}{\end{definition}}
\newtheorem{example}[thm]{Example}
\newenvironment{exmp}{\begin{example}\rm}{\end{example}}
\newtheorem{remark}[thm]{Remark}
\newenvironment{rem}{\begin{remark}\rm}{\end{remark}}
\newtheorem{protocol}[thm]{Protocol}
\newenvironment{prot}{\begin{protocol}\rm}{\end{protocol}}
%\newcommand{\mod}{\, {\rm mod}\,}
\newcommand{\A}{\mathcal{A}}
\newcommand{\B}{\mathcal{B}}
\newcommand{\C}{\mathcal{C}}
\newcommand{\D}{\mathcal{D}}
\newcommand{\M}{\mathcal{M}}
\newcommand{\G}{\mathcal{G}}
\newcommand{\F}{\mathbb{F}}
\newcommand{\K}{\mathbb{K}}
\newcommand{\N}{\mathbb{N}}
\newcommand{\Z}{\mathbb{Z}}
\newcommand{\Q}{\mathbb{Q}}
\newcommand{\mat}{M\! at_{n\times n}}
\newcommand{\Mat}{\mathrm{Mat}}
\newcommand{\GL}{\mathrm{GL}}
\newcommand{\rank}{\mathrm{rank}\,}
\newcommand{\im}{\mathrm{im}\,}
\newcommand{\lcm}{\mathrm{lcm}\,}
\newcommand{\rfb}{\mathcal{R}}
\newcommand{\afb}{\mathcal{A}}
\newcommand{\p}{\mathfrak{p}}

\newcommand{\eqr}[1]{~\mbox{$(${\rm \ref{#1}}$)$}}
\newcommand{\md}[1]{\,\,\, ( \mathrm{mod}\, #1 \,) }
\newcommand{\ideal}[1]{{\langle #1 \rangle}}
\newcommand{\vier}[4]{\left[ \begin{array}{ccc}
                   #1 &\;& #2 \\ #3 &\;& #4 \end{array} \right]}
%\newcommand{\Section}[1]{\section{#1}}
\newcommand{\Section}[1]{\section{#1}\setcounter{equation}{0}}
\renewcommand{\theequation}{\thesection.\arabic{equation}}

\newcommand{\noproof}{\hspace*{1cm}\hspace*{\fill}\openbox
  \medskip\endtrivlist}

%%% AMS math stuff.
\newcommand{\openbox}{\leavevmode
  \hbox to.77778em{%
    \hfil\vrule
  \vbox to.675em{\hrule width.6em\vfil\hrule}%
  \vrule\hfil}} 
\newcommand{\proofname}{Proof}

\newenvironment{proof}[1][\proofname]{\par\normalfont
  \trivlist\item[\hskip\labelsep\itshape #1:]\ignorespaces
  }{\hspace*{1cm}\hspace*{\fill}\openbox \medskip\endtrivlist}

\newenvironment{proofs}[1][Sketch of Proof]{\par\normalfont
  \trivlist\item[\hskip\labelsep\itshape #1:]\ignorespaces
  }{\hspace*{1cm}\hspace*{\fill}\openbox \medskip\endtrivlist}
%%% End of AMS math stuff

%\pagestyle{empty}
\title{{\tt GGNFS} Documentation
  %\thanks{} 
}

\date{Draft of \today}%
\author{Chris Monico \\
  {\small Department of Mathematics and Statistics\vspace{-2mm}}\\
  {\small Texas Tech University\vspace{-2mm}}\\
  {\small {\em e-mail:\/} cmonico@math.ttu.edu }
  }


\begin{document}\maketitle
%\thispagestyle{empty}
%\begin{abstract}
%  This is a quick introduction to the number field
%  sieve and the {\tt GGNFS} software.
%\end{abstract}

\tableofcontents

%%%%%%%%%%%%%%%%%%%%%%%%%%%%%%%%%%%%%%%%%%%%%%%%%%%%%%%%%%%%%%
\newpage
\section{Disclaimer}
  \begin{enumerate}
    \item
      Much of the info here on using {\tt GGNFS} is already outdated!
      As the software gets closer to a near-final state I will 
      come back and update it. In the meantime, the best idea
      for learning the usage would be to look at the test/ directories
      having the most recent timestamps.
    \item
      This is a very preliminary version of this document - 
      I will add the relevent references to the literature
      and try to fix typos, inaccuracies, and confusing
      discussions as I find them. It is really not ready
      to be distributed or read yet; you are only reading
      it now because, for my own convenience, I am storing it
      in the source tree. Heck - it's probably not even complete
      yet.
    \item 
      This document is not intended for people who really want
      to learn the details and many intricicies of the number field 
      sieve. It is intended to give just enough insight to be able
      to either refer to the literature and learn more, or to
      be able to run the software and have at least a vague idea
      of what it's doing.
    \item
      {\tt GGNFS} is not claimed to be lightening fast, robust
      or even reliable (yet). Before you spend any exorbitant
      amount of time trying to factor some large number, start
      with smaller ones and work your way up to find/learn
      the limitations of the software. There are much better
      implementations out there. The only reason for writing
      this one is to have a complete and open source implementation
      available. It is hoped that, as it becomes more stable,
      people will begin to make contributions and improvements
      to the code. At this time, a fast pure (or almost pure) C,
      {\tt GGNFS}-compatible lattice siever would be more than
      welcome, as well as improvements to the polynomial selection
      code. Both of these tasks require serious proficiency in
      mathematics and expertise in programming. If you are up to
      the challenge - go for it!
  \end{enumerate}

\section{Preliminaries}        \label{sec:prelim}

    Throughout this document, $N$ will denote
  the number to be factored (and $N$ is known to be composite).
  For simplicity, we will also assume $N$ is a product of two
  distinct primes $N=pq$.
  For if this is not the case, then there is likely
  some other algorithm which can more efficiently factor $N$ (for
  example, if $N$ has 120 digits and 3 prime factors, then at least
  one of them has fewer than 40 digits, and so ECM would work).
  On the other hand, if $N=p^2$, then $N$ is easily factored by
  simply computing the square root.

  As is standard, we use the notation $a\equiv b \md{N}$ to denote
  the fact that $a+\Z/N\Z$ and $b+\Z/N\Z$ are the same equivalence
  class in $\Z/N\Z$ (i.e., $a-b$ is divisible by $N$). We also use
  the less standard notation $a\mod N$ to indicate the unique remainder
  $r$ resulting from an application of the division algorithm.
  That is, it is the unique value of $r$ so that for some $q\in\Z$,
  \[
    a = qN + r, \hspace{1cm} 0\le r < N.
  \]
  Said differently, $a\mod N$ is the unique {\em nonnegative} remainder
  when $a$ is divided by $N$. We will attempt to be consistent in
  our use (and abuse) of these two notations. For the reader not familiar
  with these notions, the difference is really subtle enough that
  it can be ignored. 

  The goal of many integer factorization algorithms is to produce
  two congruent squares. That is, to find nonzero $x,y\in\Z$ so that
 \begin{equation} \label{eq:twosq}
    x^2 \equiv y^2 \md{N}.
 \end{equation}
  For we will then have
  \[
    x^2 - y^2 \equiv (x-y)(x+y) \equiv 0 \md{N}.
  \]
  From Equation \ref{eq:twosq}, we see that $x/y$ is a square root
  of 1 modulo $N$ (if $y$ is invertible modulo $N$; but if not,
  $(y,N)$ will be a proper divisor of $N$).
  By our assumption that $N=pq$, there are exactly 4 distinct square roots
  of unity, and so if such an $(x,y)$ pair is randomly chosen from the
  set 
  \[
    S = \{ (x,y) \,\vert\, x^2\equiv y^2 \md{N} \}
  \]
  there is about a 50/50 chance that $x/y$ is not
  one of 1,-1. This means that $x\not\equiv \pm y \md{N}$, and
  so the gcd $(x-y, N)$ is a proper divisor of $N$. The factorization
  methods based on this observation try to do exactly that - find a 
  (pseudo) random element of the set $S$.

  

%%%%%%%%%%%%%%%%%%%%%%%%%%%%%%%%%%%%%%%%%%%%%%%%%%%%%%%%%%%%%%%%%%%%
\section{Fermat's and Dixon's methods} \label{sec:dixon}

  This section briefly describes Dixon's method, which is the
  predecessor of the Quadratic Sieve (QS). Since
  the NFS can be largely considered a horrific generalization
  of QS, it is crucial to understand the QS before moving on
  to NFS. Briefly, Dixon's method helps motivate the QS, which
  in turn helps motivate and understand the basic concepts of the
  NFS.

  Suppose $N=9017$. We could attempt to find two congruent squares
  as follows: First compute $\sqrt{N}\approx 94.96$. We can
  thus start computing $95^2 \mod N, 96^2 \mod N,\ldots$ until
  we find that one of these numbers is again a perfect square.
  \begin{center}\begin{tabular}{|r|r|}
    \hline
    $a$ & $a^2\mod N$ \\
    \hline
    95 & 8 \\
    96 & 199 \\
    97 & 392 \\
    98 & 587 \\
    99 & 784 = $28^2$\\
    \hline
  \end{tabular}\end{center}
  The last row of this table gives us that $99^2\equiv 28^2\md{9017}$,
  and so $(99-28)(99+28)\equiv 0 \md{9017}$. We compute the gcd
  $(99-28, 9017) = (71, 9017) = 71$, and so we have found that 71 is
  a proper divisor of $N$. The cofactor is easily found by division
  and so we obtain $9017 = 71\cdot 127$.
  
  The method we just described is essentially Fermat's method for
  factoring integers. However, it is quite a bad thing to do in general.
  Observe: If $N=pq$ is odd, then we may assume $q=p+k$ for some positive
  even integer $k$. It's not hard to see that this method will require
  about $k/2$ operations before it succeeds. If $p$ and $q$ each have
  say 30 digits, then $k$ will typically still be a 30 digit number!
  That is, asymptotically, this is still a $O(\sqrt{N})$ algorithm
  in the worst case, and is no better than trial division.

  This brings us to Dixon's method. If we had done a little extra work in
  the previous computation, we could have factored $N$ a little earlier.

  \begin{center} \begin{tabular}{|r|r|}
    \hline
    $a$ & $a^2\mod N$ \\
    \hline
    95 & 8  = $(2^3)$\\
    96 & 199= $(199)$ \\
    97 & 392= $(2^3)(7^2)$ \\
    \hline
  \end{tabular}\end{center}

  We can already produce two congruent squares from the data in these
  first three rows. For notice that
  \[
    (95\cdot 97)^2 \equiv (95^2)(97^2) \equiv (2^3)(2^3\cdot 7^2) \equiv 
               (2^3\cdot 7)^2 \md{N},
  \]
  whence we obtain $198^2 \equiv 56^2 \md{N}$. The gcd computation
  $(198 - 56, 9017) = (142, 9017) = 71$ gives a proper divisor of $N$.
  
  This is perhaps a misleading example: we saved only the computation
  of two squares modulo $N$. But asymptotically, it is actually much
  better than Fermat's method if done right. The general idea is to 
  choose a {\em factor base} of small primes. We are then interested
  in only values of $a$ so that $a^2 \mod N$ factors completely over
  this factor base. That is, we need not try too hard to factor the
  numbers in the right hand column - we just use trial division with
  the primes from the factor base, and throw out anything which does
  not factor completely in this way (i.e., we could have discarded the
  row corresponding to 96 since $96^2$ did not completely factor over
  small primes). If the factor base has $B$ primes,
  then it is easy to see that we need at most $B+1$ values of $a$
  so that $a^2\mod N$ factors completely over this factor base. For
  once we have found at least $B+1$, we are guaranteed that some of
  them can be multiplied out to produce two congruent squares modulo $N$.
  

%%%%%%%%%%%%%%%%%%%%%%%%%%%%%%%%%%%%%%%%%%%%%%%%%%%%%%%%%%%%%%%%%%%%
\section{Quadratic sieve} \label{sec:qs}
  Dixon's method represents a big improvement to Fermat's method.
  However, the trial division of $a^2\mod N$ over the small primes
  is quite time consuming, since many of the $a^2\mod N$ will not
  completely factor over the factor base. The net result is that
  we will waste alot of time doing trial division on numbers which
  we will only wind up throwing out anyway.

  The point of the quadratic sieve is that we can substantially
  reduce the amount of work by employing a clever {\em sieve}.
  Loosely, a {\em sieve} to a mathematician is exactly the same thing
  it is to a cook. It is a device for efficiently separating the things
  we're interested in from the things we're not interested in.
  A common sieving device is some sort of wire mesh which will hold
  large objects and allow smaller ones to pass through.
  For example, a cook might use a sieve to remove chunks of skin and
  fat from a gravy. Or, perhaps to remove coarse clumps of flour to
  leave only the fine grains. Manufacturers of abrasives use a sieve
  to separate coarse grains of particles from fine grains
  (the finest grains become fine abrasives while the coarse ones left
  behind become coarse abrasives). The old school miners used to
  use a sieve to look for gold within the mud of some creeks and rivers
  - the sieve would be used to remove the fine particles from the mud
  leaving only moderate sized rocks which could then be inspected for
  the presence of gold.

  In the case of the QS, we will use a sieve to efficiently remove
  the ``coarse'' $a^2\mod N$ values. Specifically, we will remove
  the ones which most likely do not factor over the small primes.
  The benefit is that we perform the more expensive trial division
  step only on numbers which are more likely to be useful (we
  waste much less time trying to factor numbers which aren't going
  to factor over the factor base anyway).

  The key observation that makes this possible is as follows: If $p$
  is a small prime dividing $a_0^2 - N$ for some particular 
  integer $a_0$, then
  \[
    0 \equiv a_0^2 - N \equiv a_0^2 -N + 2p + p^2 \equiv (a_0 + p)^2 -N \md{p}.
  \]
  That is, if $a_0^2 - N$ is divisible by $p$, then so is $(a_0+p)^2 - N$.

  The astute reader might have noticed that we went from talking about
  $a^2\mod N$ to $a^2 - N$. Why? Recall that in Dixon's method we started
  with an initial value of $a=a_0=\lceil\sqrt{N}\rceil$. For this value
  of $a$ and at least several after it, $a^2\mod N = a^2-N$. In fact,
  this continues to hold for values of $a$ until $a_1^2 \ge 2N$. But this
  means $a_1 \ge \sqrt{2N}$. Since our goal is to have an algorithm which
  is much better than trial division, we have no intention of performing
  anywhere near $\sqrt{N}$ operations, so {\em for all values of $a$
  in which we're interested, we can assume $a^2\mod N = a^2 - N$}. This is
  also convenient since subtraction is an easier operation than 
  division-and-take-remainder anyway.

  \begin{defn} \label{defn:smooth}
    Let $B$ be a positive integer. An integer $n$ is said to be
    {\em $B$-smooth} if all prime divisors of $n$ are less than
    or equal to $B$.
  \end{defn}

  So here is the essence of the quadratic sieve:
  \begin{enumerate}
    \item
      Choose a positive integer $B$ and
      a factor base of small primes upto $B$. Let $s$ denote
      number of primes in the factor base.
    \item
      Set $a_0\gets \lceil \sqrt{N}\rceil$. Set $a_1 \ll \sqrt{2N}$ to
      be a large enough integer that there are at least, say $s+5$,
      $B$-smooth values of $a^2-N$ with $a_0\le a\le a_1$.
    \item
      Initialize an array of size $a_1-a_0+1$.
    \item \label{qs:step4}
      For each prime $p_j$ in the factor base, do as follows:
      \begin{itemize}
        \item
          Find the smallest value of $k\ge 0$ so that 
          $(a_0 + k)^2 - N \equiv 0 \md{p_j}$ (if one exists; see
          the note below).
        \item
          While $a_0+k \le a_1$, note in the array entry corresponding to $a_0+k$
          that $(a_0+k)^2-N$ is divisible by $p_j$, then do $k\gets k+p_j$.
      \end{itemize}
    \item
      Scan through the array for entries which are divisible by many
      primes in the factor base. Do trial division on the $a^2 - N$
      corresponding to each such entry, keeping only the ones which
      are $B$-smooth.
    \item
      If there were fewer than $s+5$ factorizations kept in the previous
      step, $a_1$ was too small - repeat the same procedure over the 
      interval $[a_1+1, a_1+m]$ for a sufficiently large $m$ to produce
      more.
    \item
      Find a subset of the factorizations that can be multiplied out
      to produce two congruent squares, $x^2\equiv y^2\md{N}$.
      If $(x-y, N)=1$, try a different subset of the factorizations.
      (It is easily seen that if we have at least $s+5$ good factorizations
      over the $s$ primes, then there are at least $16$ different ways
      to produce the congruent squares. One of them will do the job
      with high probability).
  \end{enumerate}

  The important difference between QS and Dixon's method comes in Step
  \ref{qs:step4}. This is the {\em sieving step}. The point is that
  we must do a little work to find some value of $k$ so that
  $(a_0 + k)^2 - N \equiv 0 \md{p_j}$. But having done that, we get
  (almost) for free, all values of $a$ in the interval $[a_0, a_1]$
  which are divisible by $p_j$.

  There are several important points we've neglected to keep the above
  description simple: First of all, the reader may have noticed that
  the equation $x^2 - N\equiv 0\md{p_j}$ in Step \ref{qs:step4} actually
  has {\em two} distinct solutions modulo $p_j$ in general. This is
  true, and in fact, we should use them both! Also, it is not hard to
  see that there are certain primes which can be omitted from the
  factor base - those for which $\left(\frac{N}{p_j}\right)=-1$ can
  be trivially omitted, since there will be no solution for $k$ in
  Step \ref{qs:step4}. Also, we haven't mentioned anything about how
  to choose $B$; on the one hand, if it's too small, it will be very
  hard to find $B$-smooth values of $a^2-N$. On the other hand, if it's
  too large, we will need to find way too many of them! Also, we need
  not construct a really enormous array and try to find them all at once -
  in practice, one partitions the very large array into pieces small
  enough to fit into the CPU cache, and does each piece separately.
  We also did not specify exactly how to ``note in the array that
  $(a_0 +k)^2 - N$ is divisible by $p_j$. This can be done, for example, by
  initializing the array entries to zero and simply adding $\log p_j$
  to the corresponding entry. Then, at the end of the day, if an entry
  of the array is $x$, we will know that it has a divisor of size at least
  $e^x$ which is $B$-smooth.

  There are many significant improvements to the algorithm that we won't
  cover here. Indeed, the above description is only enough to implement
  a very rudimentary version of QS which could barely handle 40 digit
  numbers. With the improvements (multiple-polynomials, large prime variations,
  Peter Montgomery's block Lanczos algorithm,...) it is possible to factor 
  numbers of upto say 130+ digits with quadratic sieving.  

  The most important point is that the overview of the algorithm is
  very similar to the overall view of the NFS. It can be viewed as having
  four major steps:
  \begin{enumerate}
    \item
      Parameter selection.
    \item
      Sieving (produce relations).
    \item
      Find dependencies among relations.
    \item
      Produce final factorization.
  \end{enumerate}
  The NFS will proceed in essentially the same way - only the specifics for
  each step will change.

%%%%%%%%%%%%%%%%%%%%%%%%%%%%%%%%%%%%%%%%%%%%%%%%%%%%%%%%%%%%%%%%%%%%
\section{About the number field sieve} \label{sec:aboutnfs}

  The number field sieve in its current form is a very complicated beast.
  John Pollard had the interesting idea to (re-)factor the seventh
  Fermat number $F_7=2^{2^7} + 1$ by doing some very clever computations
  in the number field $\Q(\alpha) = \Q[x]/\ideal{x^3+2}$. It was clear
  immediately that his idea could be used to factor any number
  of the form $r^e \pm s$ for small $e$ and $s$. Subsequently, many
  people contributed to making the basic algorithm work for integers
  without such a special form. 

  The number field sieve is still more efficient for numbers of the 
  special form that Pollard first considered. However, such numbers
  can be factored with the same algorithm - only the first step becomes
  much easier and the algorithm runs with much `better parameters' than
  for more general numbers. However, factoring such numbers is still
  referred to as `Special Number Field Sieving (SNFS)' to highlight the
  fact that the factorization should proceed much more quickly than for
  arbitrary numbers of the same size. From an aesthetic point of view, it
  is very pleasing that the NFS should factor special-form numbers faster
  than more general numbers. After all, from a complexity-theoretic point
  of view, these numbers can be passed to an algorithm with a smaller
  input size. That is to say, in some sense, such numbers have lower
  (Kolmogorov) complexity since they admit a shorter description, and so
  it is aesthetically pleasing (to me, at least) that the NFS can take
  advantage of this.

  Similarly, the number field sieve applied to numbers which have no
  such special form is sometimes referred to as the General Number Field
  Sieve (GNFS) to highlight the fact that it is being applied to a
  general number. But again - the algorithm proceeds in exactly the same
  way after the first step.

  The quadratic sieve was a huge leap forward in factoring integers. 
  However, it does have something of a bottleneck - the values 
  $a^2 -N$ which are required to be $B$-smooth grow quadratically
  as $a$ increases. They very quickly become so large that they are
  less and less likely to be $B$-smooth. This problem is somewhat
  alleviated by the multiple polynomial version of the quadratic sieve
  (MPQS), but it is still the major hindrance to factoring larger
  numbers.

  The number field sieve gains its (asymptotic) efficiency over QS by
  moving it's search for smooth numbers to a set of numbers which
  tend to be much smaller. This does come at a price - the numbers in
  question must actually be simultaneously smooth over two different factor bases:
  a {\em rational factor base (RFB)} and an {\em algebraic factor base (AFB)}.
  Even still, it turns out that the gain is significant enough to make NFS
  asymptotically faster (already faster by 120 digits or so - perhaps
  lower or higher, depending on implementations).

  It is not our intent to give here a complete and thorough description
  of the algorithm. For that, the reader is referred to the many papers
  on the subject. Instead, we wish to give just an essential overview
  of the major steps and a basic insight as to how each is performed.

%%%%%%%%%%%%%%%%%%%%%%%%%%%%%%%%%%%%%%%%%%%%%%%%%%%%%%%%%%%%%%%%%%%%
\section{NFS overview} \label{sec:nfsoverview}

  This section is for the mathematically sophisticated, and may
  be skipped if desired. The next section will still be understandable
  for the purposes of running the {\tt GGNFS} code - all that will
  be lost by skipping this section is any hint as to how it actually
  works and what it really does. The number field sieve is, however,
  a rather deep algorithm and difficult to describe precisely.
  This section is intended to be just a broad overview; perhaps
  I will add another section to this document in the future
  containing a more precise description. The problem is that
  there are so many details to be filled in that a very precise
  description could easily fill ten pages and look more like
  computer code than mathematics.

  As usual, suppose $N$ is the composite number to be factored.
  Let $c_0+c_1x+\cdots c_dx^d = f(x)\in\Z[x]$ be a 
  (not necessarily monic) polynomial of degree $d$
  so that $f(m)=cN$ for some small nonzero integer $c$ and some
  $m\in\Z$. We assume $f$ to be irreducible - for in the unlikely
  scenario where $f=gh$ splits, then we would have 
  $g(m)h(m)\equiv 0\md{N}$ likely giving a factorization of $N$.

  Set $\K = \Q(\alpha) = \Q[x]/\ideal{f}$, where $f(\alpha)=0$.
  If we let $\hat{\alpha}=c_d\alpha$ there
  is a ring homomorphism from the order $\Z[\hat{\alpha}]$ to $\Z/N\Z$
  given by
  \begin{eqnarray}
    \psi : \Z[\hat{\alpha}] & \longrightarrow & \Z/N\Z \\
           h(\hat{\alpha}) & \longmapsto & h(m)
  \end{eqnarray}

  In fact, either this homomorphism extends to a homomorphism
  from the full ring of integers $\Z_\K$, or we can factor
  $N$ by simply computing $\Z_\K$. Usually the former case is
  true (in fact, {\tt GGNFS} does not even check for the latter
  case as it is extremely unlikely for numbers $N$ with even 40
  digits). The reason for taking $\hat{alpha}$ instead of $\alpha$
  is quite simply that, in general, $\alpha$ will not be an algebraic
  integer, and later computations are most easily done with respect
  to an integral basis - so we can avoid fractional arithmetic
  by considering $\hat{\alpha}$ instead of $\alpha$.
  But, of course, $\Q(\alpha)=\Q(\hat{\alpha})$, and so it doesn't
  matter which we use in principle.



  Choose bounds $B_1, B_2 > 0$ and let $\rfb$ denote the set
  of prime integers below $B_1$. $\rfb$ is called the 
  {\em Rational Factor Base (RFB)}. Recall that an ideal
  in the ring of integers $\p\subseteq \Z_\K$ is called
  a {\em first degree prime ideal} if $\p$ is a prime
  ideal with norm $N(\p) = \vert \Z_\K/\p \vert = p$ for some
  prime integer $p$. The point is that an arbitrary prime
  ideal may have norm $p^e$ for any $1\le e\le d$, but the
  first degree prime ideals have norm $p^1$. The 
  {\em Algebraic Factor Base (AFB)} $\afb$ will consist of
  all first degree prime ideals in $\Z_\K$ with norm at most
  $B_2$. It is easy to see that, with at most finitely many exceptions
  (in fact, a very small number of exceptions), these ideals
  may be represented uniquely as the set of pairs of integers
  $(p,r)$ with $p\le B_2$ a prime and $f(r)\equiv 0\md{p}$
  (to see this, consider the canonical projection from
  $\Z_\K$ to $\Z_\K/\p \cong \Z/p\Z$).
  Explicitly, we represent these internally as
  \begin{eqnarray}
    \rfb &=& \{ (p,r)\in\Z \,\vert\, 2\le p \le B_1 \mbox{ is prime}, r=m\mod p \}\\
    \afb &=& \{ (p,r)\in \Z \,\vert\, 2\le p \le B_1 \mbox{ is prime}, f(r)\equiv 0\md{p} \}
  \end{eqnarray}
  The reason for keeping track of $m\mod p$ for the primes in the RFB will
  become clear in a moment; it will allow us to sieve identically with both 
  factor bases (it also saves some computation during sieving).

  The goal is to find a set of pairs $\{(a_1, b_1),\ldots ,(a_s, b_s)\}$
  such that
  \begin{enumerate}
    \item
      $\prod(a_i - b_im)\in\Z$ is a perfect square, say $x^2$.
    \item
      $\prod(a_i - b_i\alpha)\in\Z_\K$ is a perfect square, say $\beta^2$.
  \end{enumerate}
  For if we can find such $(a,b)$ pairs satisfying this, we then have
  \[
    x^2 = \prod(a_i - b_im) \equiv \prod \psi(a_i-b_i\alpha)
                            \equiv \psi(\prod(a_i - b_i\alpha))
                            \equiv \psi(\beta)^2 \md{N},
  \]
  the congruent squares we seek.

  The way we will find such $(a,b)$ pairs is similar to the way we
  did it in the quadratic sieve: we will find many such pairs that
  are smooth over the factor bases and then find a subset which 
  gives the necessary perfect square products. On the rational
  side, this is a straightforward operation, facilitated by the
  fact that the size of the numbers involved is much smaller than
  for the QS (about $O(N^{1/(d+1)}$).

  The situation on the algebraic side is somewhat different, complicated
  by the fact that neither $\Z_\K$ nor $\Z[\alpha]$ are UFDs
  in general. Without unique factorization, the entire factor base
  idea seems to fall apart. On the other hand, $\Z_\K$ is a 
  Dedekind domain and so we have unique factorization of ideals!
  This is the fact that will save us. To be now a little more precise
  than above, we will find a set of pairs  $\{(a_1, b_1),\ldots ,(a_s, b_s)\}$
  such that
  \begin{enumerate}
    \item
      $\prod(a_i - b_im)\in\Z$ is a perfect square, say $x^2$.
    \item
      $\prod\ideal{a_i - b_i\alpha}\subset\Z_\K$ is a perfect square ideal, 
      say $I^2$.
    \item
      The product $\prod(a_i-b_i\alpha)$ is a quadratic residue 
      in all of the fields $\Z[x]/\ideal{q, f(x)}$
      for a fixed set of large primes $q$ not occurring anywhere
      else, and for which $f(x)$ is still irreducible mod $q$.
      This base of primes is called the Quadratic Character
      Base (QCB). It consists of a relatively small number of 
      primes (60 in {\tt GGNFS}).
  \end{enumerate}
  The idea is that conditions 2 and 3 together make it extremely
  likely that, in fact, 
  $\prod(a_i - b_i\alpha)\in\Z_\K$ is a perfect square.
  Loosely, it can be thought of this way: if the product is a perfect
  square, it will necessarily be a quadratic residue modulo
  in the specified fields. If it is not a perfect square, it
  will be a quadratic residue with about a 50/50 chance. So, if
  it is a quadratic residue modulo sufficiently many primes, it
  becomes very likely that it is itself a perfect square.

  Let us now describe the procedure by which we find such $(a,b)$
  pairs. We will again use a sieving procedure. There are several
  types of sieves, but we will discuss only the classical sieve
  here.

  First observe that if $(p,r)\in\rfb$, then $p\vert (a-bm)$ iff
  $a\equiv bm \equiv br \md{p}$. This is one reason for storing the RFB
  in the way we did, having precomputed $r=m\mod p$. More importantly,
  the required congruence has exactly the same form as on the
  algebraic side: if $\p$ is the first degree prime ideal corresponding
  to the pair $(p,r)\in\afb$, then $\p \vert \ideal{a-b\alpha}$
  iff $a\equiv br\md{p}$ (this follows by considering the image
  of $\ideal{a-b\alpha}$ under the canonical projection
  $\pi_\p : \Z_\K \longrightarrow \Z_\K/\p$).

  Finally we remark that we need consider only $(a,b)$ pairs such
  that $a$ and $b$ are coprime. The reason for this will be clear
  when we discuss finding dependencies shortly. Briefly, if
  we have two $(a,b)$ pairs with one a multiple of the other,
  and both appearing in the subset which generates the final squares,
  then we could remove both and still have a square. That is,
  we can do the job of both of them using at most one of them - 
  so for simplicity, we can simply ignore such multiples.

  The facts above suggest the classical sieve:
  \begin{enumerate}
    \item
      Fix an integer $b>0$, and a range of $a$ values
      $[a_0, a_1]$ (typically chosen symmetrically, $[-A, A]$).
      We will produce values of $a_0\le a\le a_1$ so that
      $(a-bm)$ is smooth over the RFB and $\ideal{a-b\alpha}$
      is smooth over the AFB.
    \item
      Initialize an array of size $a_1-a_0+1$.
    \item (rational sieve)
      For each $(p,r)\in\rfb$ do as follows:
      \begin{itemize}
        \item
          Find the least nonnegative value of $k$ so that
          $a_0+k\equiv rb\md{p}$.
        \item
          While $a_0+k \le a_1$ do as follows:
            note in the $k$-th entry of the array that
            $(a-bm)$ is divisible by (the prime corresponding to)
            the pair $(p,r)$, then $k\gets k+p$.
      \end{itemize}
    \item
      Scan through the array and indicate elements which
      are not divisible by enough primes from $\rfb$ to
      be smooth. For the elements which are (or could be)
      smooth over $\rfb$, initialize the corresponding entries
      for the algebraic sieve.
    \item (algebraic sieve)
      For each $(p,r)\in\afb$ do as follows:
      \begin{itemize}
        \item
          Find the least nonnegative value of $k$ so that
          $a_0+k\equiv rb\md{p}$.
        \item
          While $a_0+k \le a_1$ do as follows:
            note in the $k$-th entry of the array that
            $(a-bm)$ is divisible by (the prime corresponding to)
            the pair $(p,r)$, then $k\gets k+p$.
      \end{itemize}
    \item
      Output any pairs $(a,b)$ with $a$ and $b$ coprime and which
      were divisible by sufficiently many primes from both
      factor bases to be (probably) smooth over both FB's.
  \end{enumerate} 

  To determine whether or not entries are ``divisible by enough
  primes to be smooth'', we do as we did for the QS. Initialize
  the entries to zero. Then when an entry is divisible by some
  prime $(p,r)$, we simply add $\log p$ to the entry. After
  one such sieve, then, if an entry contains the value $x$,
  we know that it has at least $e^x$ prime divisors from
  the corresponding factor base. It is clear that such a procedure
  will necessarily miss some values of $(a,b)$ which are smooth
  (i.e., perhaps because it is divisible by a moderate power
  of one of the primes). For this reason, we allow for `fudge
  factors' in both sieves. We may thus wind up missing some
  smooth $(a,b)$ pairs and outputting some which are not smooth;
  so long as we generate many pairs which are smooth and not too
  many `false reports', the situation is good.

  \begin{rem}
    The code has been modified, as of version 0.33, so that the
    siever now does the bulk of the factoring of relations. Thus,
    it outputs only good relations and the number it reports as
    {\tt total: xxxxx} is an accurate total number of good relations
    found.
  \end{rem}
  
  Okay - assume we have generated many coprime pairs $(a,b)$
  which are simultaneously smooth over $\rfb$ and $\afb$. 
  Our goal is to find a subset so that
  \begin{enumerate}
    \item
      $\prod(a_i - b_im)\in\Z$ is a perfect square, say $x^2$.
    \item
      $\prod\ideal{a_i - b_i\alpha}\subset\Z_\K$ is a perfect square ideal, 
      say $I^2$.
    \item
      The product $\prod(a_i-b_i\alpha)$ is a quadratic residue 
      in all of the fields $\Z[x]/\ideal{q, f(x)}$
      for a fixed set of large primes $q$ not occurring anywhere
      else, and for which $f(x)$ is still irreducible mod $q$.
      This base of primes is called the Quadratic Character
      Base (QCB). It consists of a relatively small number of 
      primes (60 in {\tt GGNFS}).
  \end{enumerate}

  We accomplish this with $\F_2$-linear algebra. Construct a matrix
  over $\F_2$ with one column for each $(a,b)$ pair and one
  row for each element of $\rfb, \afb$, the QCB, and one additional
  row. Record in each entry, as appropriate, one of the following:
  \begin{itemize}
    \item
      The parity of the exponent of $(p,r)\in\rfb$ dividing 
      $(a-bm)$.
    \item
      The parity of the exponent of $(p,r)\in\afb$ dividing 
      $\ideal{a-b\alpha}$.
    \item 
      A 1 if the $(a,b)$ pair corresponds to a quadratic nonresidue,
      0 if it corresponds to a quadratic residue.
    \item
      A 1 if $a-bm<0$ (for moderate sized numbers, $a-bm$ will always
      be negative).
  \end{itemize}
  Then find some vectors in the kernel of the corresponding
  matrix. Hence the name of the step: the {\em matrix step}
  or {\em dependency step}.
  Such vectors correspond to a subset of the $(a,b)$ pairs
  which satisfy the necessary conditions to produce squares.
  (The last row is needed to insure that the final product is
  positive; without it, we could wind up with something like
  $-(5^{12})(7^6)(11^6)(13^4)$ for example, which has all prime
  factors with even exponents and yet is not a perfect square).

  So far, we haven't talked about particular numbers. The factor
  bases for moderate sized numbers may easily have 100,000 elements
  each. In such a case the matrix involved may be about 
  $200,000\times 201,000$ in which case Gaussian elimination
  becomes quite difficult. The situation is even worse for
  larger numbers where the factor bases may have over a million
  elements. Luckily, these matrices are sparse and Peter Montgomery
  gave a terrific block Lanczos algorithm which works over $\F_2$.


  Finally there is one last problem which is not self-evident
  until a close inspection. Once we have such a dependency in hand
  we have a set $S=\{(a_1,b_1),\ldots , (a_s, b_s)\}$ so that
  the products $\prod_{(a,b)\in S}(a-bm)\in\Z$ and
  $\prod_{(a,b)\in S}(a-b\alpha)\in\Z_\K$ are both (extremely
  likely to be) squares. Let's say the first is $x^2\in\Z$ and the
  latter is $\beta^2\in\Z_\K$. 
  Both $x$ and $\beta$ will
  be extremely large: if the factor bases have say 100,000
  entries each, then we can expect that the subset $S$ will
  contain roughly half of all the pairs. Since we have more than
  200,000 $(a,b)$ pairs in total, we expect that $S$ contains
  about 100,000 $(a,b)$ pairs. If each value of $a$ and $b$ has
  a size of only 5 digits each, the resulting $x$ can be expected
  to have about 500,000 digits. The situation is even worse for
  $\beta$ since, considered as an equivalence class in
  $\Q[x]/\ideal{f}$, the degree of $\beta$ remains bounded by $\deg(f)$
  but some amount of coefficient explosion can be expected.

  At the end of the day, it suffices to know $x\pm \psi(\beta) \mod N$
  to compute the resulting divisor of $N$. For this, it suffices
  to find $x\mod N$ and $\psi(\beta)N$. We can certainly find
  $x\mod N$ quite easily since we know it's factorization over
  $\Z$ - just multiply out the factors modulo $N$. The situation
  for computing $\psi(\beta)$, however, is decidedly more difficult.
  Recall that we do {\em not} actually know a prime factorization
  of $\beta$ (indeed, $\beta$ does not even have a unique prime
  factorization in general!). Rather, we know a prime factorization
  of the ideal $\ideal{\beta}\subset\Z_\K$. Furthermore, these
  ideals are not even principal, so there is no real hope of
  solving this problem in any straightforward way. This was one
  of the major stumbling blocks early on to generalizing the
  number field sieve to the general case. There were several
  early solutions to this problem which, while they did work
  in feasible time for modestly-sized numbers, they actually
  dominated the asymptotic runtime of the NFS.
  Luckily, Peter Montgomery solved this problem and
  gave a very fast way to compute the square root by successive
  approximations. The idea is to choose some element 
  $\delta\in\Z_\K$ divisible by many of the primes in the
  factorization of $\ideal{\beta}$. Then, do 
  $\ideal{\beta}\gets\ideal{\beta}/\ideal{\delta}$ (working with fractional
  ideals, of course). Repeating this process, we shrink 
  $N(\ideal{\beta})$ down to something manageable. Meanwhile, we
  have precomputed $\prod_{(a,b)\in S}(a-b\alpha)$ modulo some
  moderate sized primes, and we perform the same operations
  on the residue classes. If we are very careful, we wind up with
  a final $\ideal{\beta}$ of very small norm and we can use CRT
  to lift the final result from the residue classes.
  It is not quite
  this easy - there are many subtleties along the way; for example,
  it would be quite possible to wind up with a final ideal of 
  $\ideal{1}$, and yet the resulting element to be lifted still
  has extremely large coefficients. These problems are all solved
  very satisfactorily. The overall algorithm relies heavily 
  on the LLL algorithm for computing reduced lattice basis vectors,
  and so it is quite efficient in practice.
  
%%%%%%%%%%%%%%%%%%%%%%%%%%%%%%%%%%%%%%%%%%%%%%%%%%%%%%%%%
\section{Overview of {\tt GGNFS}}
  To factor a number with {\tt GGNFS}, there are five major 
  steps:
  \begin{enumerate}
    \item
      Select a polynomial. This is done with either the
      program {\tt polyselect} or by hand (for special form
      numbers, like $2^{2^{9}}+1$).
    \item
      Build the factor bases. Obviously this is done with 
      the program {\tt makefb}. 
    \item
      Sieve. Classical sieving (slow) is done with the program
      {\tt sieve}. Lattice sieving (faster)
      is done with {\tt gnfs-lasieve4I12e} or similar (these
      are Franke's programs, included in {\tt GGNFS} for simplicity).
    \item
      Process relations. This is done with the program {\tt procrels}.
      It takes siever output, filters out duplicates and unusable
      relations. If there might be enough relations, it will
      attempt to combine partial relations into full relation-sets
      to build the final matrix. If this happens, the program will
      tell you that you can proceed to the matrix step. If not,
      go back and do more sieving.
    \item
      Find dependencies (aka, the matrix step). This is done
      with the program {\tt matsolve}. It will take the matrix
      spit out in the previous step and find vectors in the kernel.
      First, though, the program will attempt to shrink the matrix
      a bit by removing unnecessary columns and rows, and doing
      some clever column operations to further shrink the matrix.
      The more overdetermined the matrix is, the more luck the
      pruning step will have with shrinking the matrix. Thus, it
      is sometimes preferable to do a little more sieving than
      really needed to get a larger matrix than really needed,
      and let matsolve spend some time shrinking it.
    \item
      Compute the final factorization. This is done by the program
      {\tt sqrt}.
  \end{enumerate}
  Each of these programs depends on the output of the previous program.
  The {\tt sqrt} program can only be run when the {\tt matsolve}
  program reports that it did solve the matrix and find dependencies.
  In turn, the {\tt matsolve} program can only be run when the
  {\tt procrels} program tells you to do so.

  Each step requires separate explanation.

  If you have no idea at all what these steps mean, fear not - 
  there is a Perl script included which will take care of everything
  for general numbers. For special form numbers, you will have to
  choose a polynomial by hand. (As of version 0.70.2, the script can
  do polynomial selection for numbers of certain sizes - the default
  parameter file still needs to be filled more, though, before this will
  work well).

%%%%%%%%%%%%%%%%%%%%%%%%%%%%%%%%%%%%%%%%%%%%%%%%%%%%%%%%%
\section{Polynomial selection (special numbers)}
  We need to construct a polynomial $f(x)$ with integer coefficients
  so that $f(m)=cN$ for some integer $m$ and some small nonzero integer
  $c$. The degree of $f$ should usually be 4 or 5, and we should try
  to have $f$ monic, if possible. We will say more about this later.

  We will proceed now by example, and hope the reader can figure out
  the general case. Suppose we wish to factor the following number \\
  $N =$ {\small 20442610510261172432934474696563682085166033518569276933812702775231664587063323028257425395652632767} \\
  which is a c132 divisor of $2\cdot 10^{142}-1$. (The typical situation
  is that one wants to factor a number like $2\cdot 10^{142}-1$,
  and after removing a few small divisors, there is still a relatively
  large divisor left over, like this c132).

  We would like to find an $f(x)$ with degree 5. The most straightforward
  way to do this is as follows: notice that $N$ is also a divisor of the
  number
  \[
    10^3(2\cdot 10^{142}-1) = 2\cdot 10^{145}-1000 = 2\cdot m^5 - 1000,
  \]
  where $m=10^{29}$. Thus, we {\em could} take $f(x)=2\cdot x^5 -1000$
  and we would have that $N$ is a divisor of $f(m)$ as needed. However,
  this is a silly choice, and indeed {\tt GGNFS} won't let us do it.
  The reason is quite simple: we have presumably removed all the small
  factors from $N$ - in particular, we have already removed any factors
  of 2. However, $2\cdot m^5 - 1000$ is trivially divisible by 2, so we
  can easily remove this factor and still have that $N$ is a divisor
  of $m^5-500$. Thus, we should choose $f(x)=x^5-500$.

  There are all sorts of little tricks that one can play here to construct
  polynomials for special form numbers. Several rules of thumb, though:
  \begin{itemize}
    \item
      It is preferable to have a leading coefficient of 1, if this can
      be done without enlarging other coefficients by too much.
    \item
      It is generally preferable to have the leading coefficient at least
      smaller than any others if possible.
  \end{itemize}
  To illustrate the first point, consider the number
  $N=16\cdot 10^{143}-1$.
  If we followed the same construction as above in a literal way,
  the obvious choice of polynomial would be $f(x)=4x^5-25$ with
  $m=10^{29}$. However, a better choice would probably be
  $f(x)=x^5-200$, with $m=2\cdot10^{29}$. See what we did there?
  We simply multiplied through by a 200 and noticed that
  $200N = 32\cdot 10^{145} - 200 = (2\cdot 10^{29})^5 - 200$.

  One caveat to watch out for: in the above example, we are factoring a c132.
  However, $f(m)$ is actually a number with 147 digits. Since most
  of the algorithm really uses just $f$ and $m$, the level of difficulty
  is essentially the same as if we were factoring a special c147.
  Thus, not all special form numbers are equal: we would say that this
  factorization has {\em SNFS difficulty 147} even though it is only
  a 132 digit number. The rule of thumb here is that if the SNFS difficulty
  is less than about 135\% the size of the number, it is surely worth
  doing it with a special polynomial. However, if we were factoring a 
  100 digit divisor of something like $7\cdot 10^{149}-1$, it would
  probably be preferable not to even use the special form - instead,
  it would be more efficient just to treat it as a general number.

  Finally, we haven't yet worked out optimal degree choices
  for {\tt GGNFS}, but a rough rule of thumb is probably that anything with 80-110
  digits should probably have a degree 4 polynomial, and 110-160 or so should
  have degree 5 (perhaps even a little larger: maybe 115-170 digits).

%%%%%%%%%%%%%%%%%%%%%%%%%%%%%%%%%%%%%%%%%%%%%%%%%%%%%%%%%
\section{Polynomial selection (general numbers)}

  For general numbers, use the program {\tt polyselect}
  to construct a polynomial (in fact, the advanced user may
  want to find several candidates and do some sieving experiments
  to choose one).

  To illustrate, consider the following file for the number
  RSA-100 (it is distributed with {\tt GGNFS} as
  \verb~<GGNFS>/tests/rsa100/polysel/rsa100.n~).
  \begin{verbatim} 
  name: rsa100
  n: 1522605027922533360535618378132637429718068114961380688657908494580122963258952897654000350692006139
  deg: 5
  bf: best.poly
  maxs1: 58
  maxskew: 1500
  enum: 8
  e0: 1
  e1: 1000000000
  cutoff: 1.5e-3
  examinefrac: 0.25
  j0: 200
  j1: 15
  \end{verbatim}

  The `name' field is optional, but useful. Give your number a descriptive
  name. The `n' field is obviously the number we want to factor and `deg'
  specifies the degree of the polynomial to search for. In this range of
  100 digits, we are right around the break-point between wanting degree
  4 and degree 5, but degree 5 seems to be a better choice for most numbers.
  And degree 5 will suffice for anything upto at least 155 digits or so.

  The `maxs1' figure is used as a filter. The program will go through many
  choices of polynomials, performing inexpensive tests first. Any polynomials
  which pass through `stage 1' will move on and be subjected to more expensive
  tests. Lower values mean fewer polynomials will make it through stage 1,
  and so the program will perform the expensive tests on fewer polynomials.
  Be careful not to abuse this option: while it is true that any polynomials
  which very high at stage 1 will not make good candidates, it is not
  true that a lower stage 1 score translates to a better polynomial. So this
  is used just to skip over polynomials which are obviously poor. The choice
  of 58 here seems to work well for numbers around 100 digits, but you will
  have to increase this if you're searching for a polynomial for a larger number.
  For example, for 130 digits, {\tt maxs1: 72} seems to be a better choice.
  Expect to have to do some experimentation here to find what seems to be a good
  choice (or, ask for advice from someone other than me who's already done a
  factorization with GGNFS at around the same size). When more factorizations
  have been completed, a table of reasonable values will be made available.

  The `maxskew' field should pretty much be somewhere around 1500 or 2000. Maybe
  even a bit larger, but don't go too crazy with this. The program will toss
  out any polynomials with a skew above this figure. Highly skewed polynomials
  are sometimes much better than lowly skewed ones, but there may be issues if
  the skew is too high.

  The 'enum' option works in conjuction with `e0' and `e1'.
  It specifies that {\tt polyselect} should look at polynomials
  whose leading coefficient is divisible by this value. Not only that, but
  it will enumerate them, looking at polynomials with leading coefficients
  $e0\cdot enum, (e0+1)enum, (e0+2)enum,\ldots, e1\cdot enum$.
  This should generally be chosen to be a product of small primes, but choose
  it as small as is reasonable. However, if I were looking for a polynomial for
  a c130, it would take too long to enumerate all interesting polynomials whose
  leading coefficients are multiples of 8. So in that case, I would probably
  choose this to be 720 or so.
  The idea is that good polynomials will generally have leading coefficients
  divisible by small primes. But there is the possibility that a much better
  polynomial may exist with one or two larger prime divisors. So choose this
  to be as small as possible, while insuring that polyselect will get through
  enough iterations to look at many polynomials whose leading coefficient
  has reasonable size (say, at least half as many digits as m, possibly more).
  Often, some size ranges of the leading coefficient seem to be much more productive
  than other size ranges (there is a reason for this). You may then choose to
  hone in on that particular range, looking at more polynomials there:
  choose a smaller value of `enum', but take values of `e0' and `e1' to start
  the program looking at leading coefficients in that range.

  The value of `cutoff' tells {\tt polyselect} to store any polynomial with a score
  higher than this. This is for the advanced user who wishes to have several
  candidate polynomials to choose from at the end of the process. If you have no
  particular desire to do this, simply take it to be 1.0. Whatever you do,
  don't choose it to be too small! If you do, you may quickly start filling your
  harddrive with a bunch of garbage polynomials! These will be appended to the
  file `all.poly', as will each new best candidate as it is found. At any given
  moment, though, the file `best.poly' will contain the single best polynomial
  found so far during the current run.

  The `examineFrac' field is another filter. It is a filter on the size of
  the 3rd coefficient, and will stop polyselect from performing expensive
  tests on any polynomials whose 3rd coefficient is too much larger than
  a certain relative level. So here, smaller means fewer polynomials will 
  have expensive tests done and so {\tt polyselect} will be able to examine
  more polynomials. But again - don't go too nuts - while it is generally
  true that any polynomials whose 3rd coefficient is much too large will
  be bad, it is not always true that smaller is better! The best choices
  for this seem to be between 0.2 and 0.4.

  A brief word about the two filtering parameters: it is often useful to
  do several runs with different values. For example, I will sometimes do
  a quick run with a low-to-medium `maxs1' value and a lower `examineFrac'
  value to filter out many candidates. This lets me do a quick run through
  many polynomials and quickly get something with a moderate score. But I
  will do this only so I have some idea what to expect score-wise. I will
  then go back and tune them back a bit to allow more candidates to pass
  through the filters and be tested.

  Finally, the `j0' and `j1' fields are exactly as in Murphy's thesis 
  \cite{Murphy1999}. After some kind-of-expensive tests, a more expensive
  sieve-over-nearby-polynomials is done to look for any ``nearby'' polynomials
  which have similar size properties, but which may have better root properties
  (more small roots, meaning more algebraic primes of small norm, and so
  ideals are slightly more likely to be algebraically smooth). These control
  the sieving area, so that larger means more sieving. However, it also means
  slower and if these are too high, we will be looking at polynomials too far
  from the given candidate, which may then have radically different size properties.
  Some experimentation still needs to be done here and I'm not entirely sure
  that this part of the code is working correctly, so just keep these figures
  in this neighborhood for now.

%%%%%%%%%%%%%%%%%%%%%%%%%%%%%%%%%%%%%%%%%%%%%%%%%%%%%%%%%
\section{Lattice Sieving}
  Franke's lattice siever is included in {\tt GGNFS} for pretty
  fast sieving. There are three binaries to choose from, depending
  on the size of the number you're factoring: 
  {\tt gnfs-lasieve4I12e, gnfs-lasieve4I13e, gnfs-lasieve4I14e}.\\
  \begin{center}\begin{tabular}{|r|l|}
    \hline
    12e & GNFS numbers upto about 110 digits (SNFS 150) \\
    \hline
    13e & GNFS numbers upto around 130-140 (SNFS 180) \\
    \hline
    14e & Larger \\
    \hline
  \end{tabular}\end{center}
  Although the 12e one will often appear faster than the others even
  at the higher ranges, you should still use the recommended one. It will
  generally complete the factorization faster even though the siever itself
  appears slower. This is because, for example, the 13e siever sieves over
  a larger area for each special-q. The net result is that more relations
  are found per special-q. In the worst case scenario, if you are finding too
  few per special-q by using too small of a siever, you will have to use too
  many different q's or even run out of worthwhile q's to use. As the q's get
  larger and larger, the time per relation slows down. Worse still, if you
  are in a q-range outside the AFB, then remember that the more relations
  you have with the same special q's, the more full relations you will likely
  get after combining partials! So, you may wind up needing fewer total relations
  with the 13e siever than with the 12e. You may even do some experiments to
  see this for yourself if you don't believe me. The in-between ranges are fuzzy,
  though, and you should use your own judgement.

  Having said that, how do you use it? The siever is actually a slightly modified
  version of Franke's lattice siever (for those who happen to be familiar
  with it). You will need a job file. Here is a sample job file:
  \begin{verbatim}
  n: 1402515759508324284044248459719407607675384658637522876102952911753572573612421171468891326786545449471
  m: 10000000000000000000000000
  c5: 79
  c4: 0
  c3: 0
  c2: 0
  c1: 0
  c0: -70
  skew: 1
  rlim: 800000
  alim: 699999
  lpbr: 25
  lpba: 25
  mfbr: 44
  mfba: 44
  rlambda: 2.2
  alambda: 2.2
  q0: 700000
  qintsize: 50000
  \end{verbatim}
  The lines down to `c0' are just the polynomial. The `skew' figure is computed
  automatically for you if you use polyselect to generate a polynomial, but
  you need to choose it by hand for special numbers. It should be at least 1,
  and will be larger depending on the approximate ratio of the size of the
  coefficients from high-order to low order. For most SNFS numbers, it will be between
  1 and 10 or so; you can take a guess, do some sieving, and try to adjust it to
  see if you get a speed increase. Do this several times until you've found a good
  choice.

    Then `rlim' is the limit for the RFB, `alim' is the limit for the AFB.
  `lpbr' is the max number of bits in a large rational prime, `lpba' the
  same on the algebraic side. `mfbr' is the max number of bits for which sieve-leftover
  stuff will be factored (i.e., to find leftover large rational primes). 
  `mfba' is the same on the algebraic side. `rlambda' describes how far from perfect
  sieve values we should actually look for good relations. It is essentially
  a fudge factor, and higher means more sieve entries will be closely examined.
  If this is too low, you'll miss too many potential good relations. If it's too
  high, the siever will spend too much time looking at locations which do not
  give a good location. Generally, good values are somewhere between 1.5 and 2.6
  (maybe even a little larger - I haven't done any very large factorizations yet,
  so I don't have enough experience to say for sure). You guessed it - `alambda'
  is the same thing on the algebraic side.

  Finally, `q0' is the first special q to be examined and `qintsize' specifies that
  the siever should sieve over all special-q in the range `q0 + qintsize'.
  The siever insists that `q0' should be greater or equal to `alim' (for sieving
  with special-q from the algebraic side; for special-q from the rational side,
  it would have to be greater or equal to `rlim').
 
  If the job file above is called `snfs.job', then a typical invocation of the
  lattice siever might be
  \begin{verbatim}
    gnfs-lasieve4I12e -k -o spairs.out -v -a snfs.job  
  \end{verbatim}
  You can look in the {\tt src/lasieve4} directory at Franke's documentation
  to see what the options are.
  After a sieve run as above, the output will be in the file `spairs.out'.

%%%%%%%%%%%%%%%%%%%%%%%%%%%%%%%%%%%%%%%%%%%%%%%%%%%%%%%%%
\section{Relation processing}
  While the lattice siever does create a factor base, it may not necessarily
  be the same size as the factor base you want (in particular if you want to
  trick it into sieving with some special-q from the AFB which you want to
  use). Thus, before the first time you do relation processing, you'll
  actually need to run `makefb' to create a factor base which the {\tt GGNFS}
  programs can use. For this, you need to give it the polynomial and other
  parameters (rlim, alim, lpbr, lpba). These values can be supplied on the
  command-line or from with the polynomial file.

  Anyway, having made a factor base, the siever output is processed by
  the program `procrels'. Such processing is needed from time to time
  to see how many full relations you have (a full relation is actually
  a relation-set, resulting from combining almost smooth relations to
  ``cancel out'' leftover large primes). The basic useage is fairly simple,
  and might look like:
  \begin{verbatim}
    procrels -fb snfs.fb -newrel spairs.out -maxrelsinff 40
  \end{verbatim}
  This will take the new relations, complete their factorizations (i.e.,
  ramified primes and such which are not actually in the AFB), filter
  out duplicates, and give you some idea how many more you need.
  If there are many relations (i.e., a few million), this may begin
  to take a noticeable time and it will surely take a noticeable amount
  of disk space to store the binary data files.

  It is completely incremental - each time you run it on an spairs.out file,
  procrels will take the relations from that file, filter out duplicates,
  complete the factorizations, and add the resulting information to it's
  binary data files. It will not have any use for the spairs.out file again.

  The final option, `-maxrelsinff 40' is telling it not to use more than
  40 relations per relation-set (I think some people call this the `merge-level',
  but I'm not sure whether or not it is exactly the same thing).
  `procrels' will tell you if there were enough full relations, and it will
  build the matrix files needed by `matsolve'.
  If you get a resulting matrix which is far too
  dense, you can lower this to 32 or even 24, but you'll have to do more
  sieving to make up the difference.

  There is probably much more I could say about relation processing, but I'll let
  it go for now. However, there is one other option which is useful from
  time to time. If, for some reason, you believe that your binary files may
  have been corrupted or something very funny has happened, you can instruct
  the program to dump all relations to file in siever-output format. You could
  then delete all binary data files and other junk, and reprocess all relations
  without having to repeat all the sieving. The command-line option for performing
  this dump is, plainly enough, `-dump'. You'll just need to also give it
  the name of the `.fb' file. It will dump relations to files spairs.dump.0,
  spairs.dump.1,... and so on, with upto 500K relations per file. Be aware
  that this can take quite alot of disk space and reprocessing all of those
  relations is going to take some time when you get to it. So try to avoid
  doing this if you can help it. But it beats having to start a factorization
  over from scratch. It also provides a means by which you can change factorization
  parameters mid-factorization (i.e., perhaps you decide to change factor
  base limits or something). 

%%%%%%%%%%%%%%%%%%%%%%%%%%%%%%%%%%%%%%%%%%%%%%%%%%%%%%%%%
\section{The matrix step}
  This is easy enough - if procrels tells you that it wrote
  such-and-such data files and ``you can now run matsolve'', then
  you are ready for this step (maybe - you might choose to 
  do a little extra sieving to get a more overdetermined matrix
  which can then be pruned down to a smaller, less dense matrix
  which will solve quicker).
  Running it is quite easy: jut run it from within the directory
  where the data are living. It will find everything on it's own,
  solve the matrix, and store the result in a canonically named file.

  There are some undocumented command-line options for the program
  which can control certain things. These will be documented in the
  future, but they're not necessary.

  The first thing matsolve will do is take advantage of the fact 
  that the matrix is (hopefully) very overdetermined to do some
  pruning. It will throw out some dense columns, throw out columns
  corresponding to rows with only one nonzero entry, toss out empty
  rows and so on. It will also do some combining of columns to reduce
  the matrix dimensions without increasing the matrix weight by too 
  much. It is a very ad-hoc procedure which works acceptably. Once
  it gets to a point where it cannot shrink the matrix any further,
  it will use Montgomery's block Lanczos algorithm \cite{Montgomery1995}
  to find 32 vectors in the kernel (in fact, it will find 64 of them,
  but 32 will be discarded since the remaining 32 will already be more
  than enough).

%%%%%%%%%%%%%%%%%%%%%%%%%%%%%%%%%%%%%%%%%%%%%%%%%%%%%%%%%
\section{The square root step}
  Okay, so if the matrix step succeeded, your almost ready to
  receive the final factorization. Sample useage of the
  sqrt program is as follows
  \begin{verbatim} 
    sqrt -fb snfs.fb -deps deps -depnum 0
  \end{verbatim}
  Obviously, snfs.fb is the name of my factor base file. The
  `deps' is simply the name of the file where matsolve records
  the dependencies by default, and `-depnum 0' indicates that
  we wish to run the program on dependency 0.

  Recall that a dependency may produce the trivial factorization
  with about a 50\% chance. If this happens, no problem:
  re-run it with `-depnum 1' to try again with dependency 1,
  and so on, until you get it. On success, the divisors will
  be output on screen as `r1=...' and `r2=...' and stored in
  the logfile, `ggnfs.log'. These will be followed by 
  {\tt (pp {\it <digits>})} or
  {\tt (c {\it <digits>})} depending on whether the given
  divisor is probable prime or composite.


%%%%%%%%%%%%%%%%%%%%%%%%%%%%%%%%%%%%%%%%%%%%%%%%%%%%%%%%%
\section{Running the script}
  Luckily for all of us, there is a Perl script which can
  perform nearly all of the above steps automagically.
  As of 0.70.2, the only thing it cannot do is to construct
  a polynomial for SNFS numbers.

 

  It can choose parameters and run all the programs in 
  order automatically. The parameter choices it makes
  may not be optimal, so you may still want to learn a good
  deal about the process so you can tweak it as you go.
  
  I'll say more later about customizing the script and its
  capabilities. For now, let's see how to run it:

  Case I: I have a polynomial file for the number I want to
  factor. It's called `snfs-c112.poly'. This is easy:
  \begin{verbatim}
    factLat.pl snfs-c112
  \end{verbatim}
  or
  \begin{verbatim}
    factLat.pl snfs-c112.poly
  \end{verbatim}

  Case II: I have a general number I want to factor. It is moderate
  sized (say, less than 110 digits), and I'm willing to accept the
  fact that the script will make sub-optimal choices and the factorization
  might take longer than it should. In this case, make a new text file with
  the single line:
  \begin{verbatim}
    n: my-number-here
  \end{verbatim}
  call it myc102.n. Then just do 
  \begin{verbatim}
    factLat.pl myc102
  \end{verbatim}
  The script will look first for a file called `myc102.poly'. When it cannot
  find it, it will assume you want the script to generate it for you, so it
  will lookup some parameter choices and launch polyselect. There is also
  a built in cap for how much time it will spend looking for a good poly,
  so just let it go. It will do everything on it's own, and eventually
  (hopefully) return to you with a factorization in a day or so (for a c100 - 
  expect several days for a c110 and maybe a week or a little over for a
  c120). Of course, the timings depend on the machine - these are roughly
  for my Athlon 2800+ laptop.



%%%%%%%%%%%%%%%%%%%%%%%%%%%%%%%%%%%%%%%%%%%%%%%%%%%%%%%%%
\begin{thebibliography}{10}

\bibitem{Bach1996}
E.~Bach and J.~Shallit.
\newblock {\em Algorithmic Number Theory, Vol. 1}.
\newblock The MIT Press, Cambridge, 1996.

\bibitem{Bernstein1993}
D.~Bernstein and A.~Lenstra.
\newblock A general number field sieve implementation.
\newblock In A.~Lenstra and H.~Lenstra, editors, {\em The development of the
  number field sieve}, Lecture Notes in Mathematics 1554. Springer-Verlag,
  1993.

\bibitem{Cohen1993}
H.~Cohen.
\newblock {\em A Course in Computational Algebraic Number Theory}.
\newblock Springer-Verlag, New York, third edition, 1993.
\newblock Corrected Printing, 1996.

\bibitem{Lang1994}
S.~Lang.
\newblock {\em Algebraic Number Theory}.
\newblock Springer-Verlag, 2 edition, 1994.

\bibitem{Lenstra1993}
A.~Lenstra and H.~Lenstra, editors.
\newblock {\em The development of the number field sieve}.
\newblock Lecture Notes in Mathematics 1554. Springer-Verlag, 1993.

\bibitem{Montgomery1994}
Peter~L. Montgomery.
\newblock Square roots of products of algebraic numbers.
\newblock In {\em Mathematics of Computation 1943--1993: a half-century of
  computational mathematics (Vancouver, BC, 1993)}, volume~48 of {\em Proc.
  Sympos. Appl. Math.}, pages 567--571. Amer. Math. Soc., Providence, RI, 1994.

\bibitem{Montgomery1995}
Peter~L. Montgomery.
\newblock A block {L}anczos algorithm for finding dependencies over {${\rm
  GF}(2)$}.
\newblock In {\em Advances in cryptology---EUROCRYPT '95 (Saint-Malo, 1995)},
  volume 921 of {\em Lecture Notes in Comput. Sci.}, pages 106--120. Springer,
  Berlin, 1995.

\bibitem{Murphy1999}
B.~Murphy.
\newblock {\em Polynomial selection for the number field sieve factorization
  algorithm}.
\newblock PhD thesis, The Australian National University, 1999.

\bibitem{Nakamula1999}
Ken Nakamula.
\newblock A survey on the number field sieve.
\newblock In {\em Number theory and its applications (Kyoto, 1997)}, volume~2
  of {\em Dev. Math.}, pages 263--272. Kluwer Acad. Publ., Dordrecht, 1999.

\bibitem{Pollard1993}
J.~Pollard.
\newblock Factoring with cubic integers.
\newblock In A.~Lenstra and H.~Lenstra, editors, {\em The development of the
  number field sieve}, Lecture Notes in Mathematics 1554. Springer-Verlag,
  1993.

\bibitem{Pollard1993a}
J.~Pollard.
\newblock The lattice sieve.
\newblock In A.~Lenstra and H.~Lenstra, editors, {\em The development of the
  number field sieve}, Lecture Notes in Mathematics 1554. Springer-Verlag,
  1993.

\end{thebibliography}
\end{document}
